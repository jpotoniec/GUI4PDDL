\chapter{Podsumowanie}

W prezentowanej pracy zaprojektowano oraz zaimplementowano graficzne narzędzie do opisu problemów planowania
w języku \emph{PDDL}.
Posiada ono funkcjonalność zintegrowanego środowiska programistycznego. Umożliwia tworzenie i zarządzanie
projektami oraz plikami \emph{PDDL}. W skład narzędzia wchodzą edytor strukturalny, moduł analizy kodu oraz moduł
integracji z zewnętrznym oprogramowaniem planisty. Wśród zaimplementowanych funkcji można wymienić: kolorowanie kodu źródłowego, znajdowanie błędów składniowych i semantycznych,
podpowiadanie składni, proste uruchamianie planisty, przeglądanie wyznaczonych planów.

Cel pracy, opisany w rozdziale 1.1, został spełniony. Zaimplementowano wszystkie funkcje, których specyfikacja
została zamieszczona w rozdziale 3.1. Automatyczne testy jednostkowe oraz ręczne testy integracyjne wykazały
poprawność i stabilność pracy oprogramowania. Wymagania pozafunkcjonalne zostały spełnione.

Narzędzie \emph{GUI4PDDL} zostało zaimplementowane jako wtyczka do środowiska Eclipse. Dzięki temu 
zespół miał możliwość zapoznania się z architekturą rozbudowanej aplikacji. Jest to cenne doświadczenie,
które może zostać wykorzystane w pracy zawodowej.

Planowane jest wykorzystanie opracowanego narzędzia na zajęciach dydaktycznych prowadzonych na Wydziale 
Informatyki Politechniki Poznańskiej w ramach przedmiotu Sztuczna Inteligencja. Opiekun niniejszej pracy mgr Jędrzej
Potoniec prowadzi zajęcia laboratoryjne z tego przedmiotu. Wprowadzenie zintegrowanego środowiska programistycznego
ułatwi studentom tworzenie i testowanie problemów planowania. Przyczyni się to do lepszego zrozumienia tematyki
automatycznego planowania.

Opracowane oprogramowanie posiada modułową strukturę. Dzięki temu w łatwy sposób może zostać dostosowane do 
nowych wymagań.

Wśród możliwości przyszłej rozbudowy narzędzia warto wymienić moduł symulacji planów, pełniący rolę narzędzia
do usuwania błędów. Moduł mógłby umożliwiać wykonywanie ciągu akcji w trybie pracy krokowej.
Możliwa byłaby wizualizacja zbioru spełnionych predykatów oraz wskazanie akcji ze spełninymi warunkami wstępnymi.
Podobne funkcje, w odniesieniu do systemów eksperckich, zawiera narzędzie \emph{CLIPS}\footnote{http://clipsrules.sourceforge.net}.

Inną możlwiością przyszłego rozwoju oprogramowania jest dodanie wsparcia dla języków wywodzących się z PDDL.
Można do nich zaliczyć \emph{Probabilistic PDDL} \cite{ppddl}, \emph{PDDL+} \cite{pddlplus} i \emph{Web-PDDL}
\cite{webpddl}. 

