\chapter{Wykorzystane oprogramowanie}
\section{Eclipse}

\section{ANTLR}
ANTLR (ang. \emph{ANother Tool for Language Recognition}) jest generatorem
analizatorów leksykalnych i składniowych.

Narzędzie przetwarza opisy gramatyk języków formalnych w postaci 
zbliżonej do notacji
 Backusa-Naura. Na podstawie specyfikacji generowany jest kod analizatora.
Językami programowania wspieranymi przez ANTLR są między innymi C, C\#, Java
i Python.

Istotną zaletą oprogramowania jest jego prostota użycia. Wygenerowany kod 
jest czytelny i łatwo można go dołączyć do własnej aplikacji. Postać plików
wejściowych jest podobna dla wszystkich rodzajów analizatorów. Dodatkowo
istnieją narzędzia wspierające projektowanie i implementację gramatyk
formalnych z wykorzystaniem ANTLR. Należy do nich ANTLRWorks, będące
zintegrowanym środowiskiem programistycznym. Zawiera ono edytor strukturalny,
zintegrowany program do usuwania błędów i moduł do wizualizacji struktur 
danych powstających podczas translacji.

ANTLR został opublikowany wraz z kodem źródłowym na licencji BSD. Projekt jest
aktywnie rozwijany. Wiele programów zawiera analizatory wygenerowane za pomocą
ANTLR. Należą do nich Hibernate, Jython i Netbeans.

\subsection{Analiza LL(*)}
Narzędzie ANTLR generuje analizatory LL(*). Służą one do przetwarzania 
języków opisanych gramatykami bezkontekstowymi LL(*).

Analizatory LL, będące poprzednikiem analizatorów LL(*) wykorzystują
analizę zstępującą i metodę zejść 
rekurencyjnych. Każdemu nieterminalowi zdefiniowanemu w specyfikacji gramatyki
odpowiada procedura lub metoda. 

Analiza, polegająca na konstrukcji drzewa
wyprowadzenia, rozpoczyna się od korzenia. Związany z nim jest symbol 
startowy gramatyki. W każdym kroku analizy wykonywane są następujące czynności:
\begin{enumerate}
\item wybierany jest pierwszy w porządku zdefiniowanym przez
  przeszukiwanie wgłąb węzeł V zawierający symbol nieterminalny A.
\item wybierana jest produkcja mająca po lewej stronie symbol A.
\item dodawane są węzły związane z symbolami znajdującymi się po prawej
  stronie produkcji z punktu 2. Stają się one dziećmi węzła V.
\end{enumerate}

Klasy analizatorów LL różnią się sposobem wyboru produkcji w punkcie 2. 
Analizatory LL(*) są analizatorami przewidującymi. Podejmują one decyzję na
podstawie nieprzetworzonych symboli leksykalnych w strumieniu wejściowym. Liczba
tych symboli nie jest ograniczona. W celu przeglądania strumienia wejściowego 
,,w przód'' tworzone są automaty skończone mogące zawierać cykle.

W odróżnieniu od tradycyjnych gramatyk LL, W gramatykach LL(*) prawe strony
produkcji dla jednego nieterminala mogą mieć ten sam prefiks. Nie ma więc 
konieczności lewostronnej faktoryzacji, co znacznie ułatwia tworzenie
gramatyk.

\subsection{Abstrakcyjne drzewa składniowe}
Przetwarzanie kodu źródłowego jest skomplikowanym procesem. W celu uproszczenia 
zadania, rozbija się je na logicznie wyodrębnione fazy. Uruchamiane są one 
sekwencyjnie. Każda z faz oblicza część informacji. 

Zwykle więcej niż jedna faza dokonuje przejrzenia wszystkich danych wejściowych,
zwanego przebiegiem. Aby uniknąć ponownej analizy leksykalnej i składniowej przy
kolejnym przebiegu, potrzebna jest pośrednia struktura danych.

Abstrakcyjne drzewo składniowe (AST) jest jedną z form reprezentacji
konstrukcji  języka. Jest to skondensowana postać drzewa
wyprowadzania, skłądająca się
wyłącznie z węzłów reprezentujących symbole wejściowe. Węzły
wewnętrzne reprezentują słowa kluczowe i operatory, liście -- argumenty.

Wszystkie fazy występujące po fazie tworzącej abstrakcyjne drzewo składniowe,
przekazują tę strukturę. Jej przetwarzanie, a tym samym działanie kolejnych
faz sprowadza się do przechodzenia drzewa.

Możliwe jest ręczne tworzenie drzewa składniowego za pomocą odpowiednich
reguł semantycznych. Takie podejście proceduralne jest opisane w [Aho].

Narzędzie ANTLR wspiera tworzenie drzew składniowych. Dodanie
odpowiedniej opcji w specyfikacji analizatora składniowego powoduje automatyczne
wygenerowanie struktury. Poza tym, specyfikacja analizatora składniowego
 może zawierać elementy deklaratywne wpływające na kształt AST.

Jednym ze sposobów przechodzenia abstrakcyjnego drzewa składniowego jest
wykorzystanie analizatora drzewowego. Jest to podprogram generowany przez
ANTLR na podstawie specyfikacji, tak jak analizator składniowy lub leksykalny.
Wprowadzenie analizatorów drzewowych wyróżnia ANTLR na tle innych generatorów
 parserów. 

\subsection{Pliki wejściowe}

Pliki wejściowe ANTLR posiadają rozszerzenie .g. Zawierają one specyfikację
analizatora leksykalnego, składniowego, drzewowego lub połączoną specyfikację
analizatorów leksykalnego i składniowego. 

Specyfikacja analizatorów składa się z listy reguł, definiujących symbole
leksykalne, sekcji opcji konfiguracyjnych oraz kodu dodawanego do generowanego
pliku źródłowego.

W przypadku plików zawierających połączoną specyfikację analizatorów, występuje
podział symboli ze względu na pierwszą literę ich nazwy.
 Symbole, których nazwa rozpoczyna się wielką literą uznawane są
za symbole terminalne. W ich definicjach mogą występować literały i inne
symbole terminalne. Na podstawie reguł definiujących symbole terminalne
tworzony jest analizator leksykalny. Pozostałe reguły tworzą analizator
składniowy.

