\chapter{Architektura systemu}
\section{Podział na projekty Eclipse}
Ze względu na złożoność oraz zakres wymagań projekt GUI4PDDL podzielono na 6 głównych podprojektów związanych z rozwojem wtyczki (\textit{Plugin Project} lub \textit{Feature Project}) oraz 4 dodatkowe, standardowe projekty Eclipse, służące do testowania istniejących rozwiązań i przechowywania zasobów. Taka struktura pozwala na logiczne oddzielenie najważniejszej części -- implementacji wtyczki od kodu, odpowiedzialnego za możliwość jej integracji oraz związanego z testami. Pozwala to zmniejszyć rozmiar plików pobieranych przy instalacji przez użytkownika, zwłaszcza dodatkowych bibliotek.

Zgodnie z konwencją nazewniczą pakietów języka Java, wszystkie główne projekty mają przedrostek \texttt{pl.poznan.put.cs.gui4pddl}, natomiast w pozostałych obowiązują dowolne nazwy. Dokładna struktura i opis poszczególnych projektów przedstawione są poniżej.

Projekty główne:
\begin{itemize}
\item \textbf{\texttt{pl.poznan.put.cs.gui4pddl}} -- projekt zawierający pełną implementację wtyczki (wersję instalacyjną). Składa się z pakietów odpowiedzialnych za logikę biznesową oraz wszystkie istniejące widoki. Szerszy opis tego projektu znajduje się w rozdziale~\ref{sec:struktura}.
\item \textbf{\texttt{pl.poznan.put.cs.gui4pddl.antlr}} -- projekt zawierający skompilowaną bibliotekę ANTLR, wykorzystywaną w przetwarzaniu gramatyki formalnej języka PDDL.
\item \textbf{\texttt{pl.poznan.put.cs.gui4pddl.feature}} -- projekt \textit{Feature}, umożliwiający połączenie różnych wtyczek w taki sposób, by z zewnątrz były traktowane jako logiczna całość, co ułatwia zarządzanie nimi. Dodatkowo pozwala na uzupełnienie informacji dotyczących wersji, licencji, obsługiwanych systemów operacyjnych itp. Projekt tego typu wymagany jest także w procesie budowania i aktualizacji.

We wtyczce GUI4PDDL grupuje podprojekty \texttt{pl.poznan.put.cs.gui4pddl}\linebreak oraz \texttt{pl.poznan.put.cs.gui4pddl.antlr}, a także udostępnia podstawowe informacje o niej.
\item \textbf{\texttt{pl.poznan.put.cs.gui4pddl.tests}} -- projekt zawierający testy jednostkowe kodu niezwiązanego z interfejsem graficznym (z wykorzystaniem biblioteki \textit{jUnit}) oraz gramatyki formalnej, generowanej przez narzędzie ANTLR, używanej w parserze (z wykorzystaniem biblioteki \textit{gUnit}). Oddzielenie testów od kodu głównego projektu pozwoliło na zmniejszenie wielkości wersji instalacyjnej oraz zredukowanie zależności od dodatkowych bibliotek.
\item \textbf{\texttt{pl.poznan.put.cs.gui4pddl.uitests}} -- projekt zawierający testy jednostkowe interfejsu graficznego, z wykorzystaniem biblioteki \textit{SWTBot}. Biblioteka ta umożliwia przeprowadzenie automatycznych testów akceptacyjnych GUI, stworzonego przy pomocy biblioteki \textit{SWT}.
\item \textbf{\texttt{pl.poznan.put.cs.gui4pddl.update}} -- projekt typu \textit{Update site}, zawierający statyczne pliki, które mogą być umieszczone w określonym miejscu na serwerze. Użytkownicy, korzystając z adresu do tego miejsca mogą pobrać i zainstalować wtyczkę poprzez menedżer aktualizacji. Projekt ten korzysta z wcześniej zdefiniowanego projektu \textit{Feature} oraz zawiera opis wtyczki i kategorię wyświetlone następnie przy instalacji.
\end{itemize}

Pozostałe projekty:
\begin{itemize}
\item \textbf{\texttt{checker}} -- ?
\item \textbf{\texttt{grammar}} -- ?
\item \textbf{\texttt{RubikCube}} -- przykładowy projekt PDDL wykorzystywany podczas rozwijania wtyczki. Zawiera zadanie automatycznego planowania, które z racji swojego charakteru oraz używanego plannera wykonywało się przez znaczny czas, co było pomocne podczas testowania przerywania procesu planowania (roz. \ref{subsec:przerywanie}).
\item \textbf{\texttt{WorldOfBlocks}} -- przykładowy projekt PDDL wykorzystywany podczas rozwijania wtyczki. Zawiera zadanie automatycznego planowania, które z racji swojego charakteru oraz używanego plannera wykonywało się przez krótki czas.
\end{itemize}

\section{Struktura wtyczki}
\label{sec:struktura}

Kod głównego projektu wtyczki GUI4PDDL (\texttt{pl.poznan.put.cs.gui4pddl}) podzielony jest pomiędzy szereg pakietów, których nazwy, zgodnie z konwencją przyjętą w języku Java wskazują na zakres odpowiedzialności. Wyróżnić można kilka odrębnych grup oraz podpakiety.

\begin{itemize}
\item \texttt{pl.poznan.put.cs.gui4pddl} -- podstawowy pakiet projektu, zawierający implementację (klasa Activator) klasy Plugin, którą musi implementować każda wtyczka platformy Eclipse. Klasa ta odpowiada za cykl życia (inicjalizacja, zakończenie) projektu. W pakiecie tym znajdują się również stałe, współdzielone pomiędzy pakietami, definicja natury PDDL (roz. \textbf{ROZDZIAŁ}), klasa magazynu obrazów (ImageCache) oraz metod pomocnych przy tworzeniu nowego projektu.
\item \texttt{pl.poznan.put.cs.gui4pddl.\textbf{codecompletion}} -- zawiera klasy odpowiedzialne za zarządzanie podpowiadaniem składni, w tym za tworzenie prawidłowych propozycji uzupełniania kodu.
\item \texttt{pl.poznan.put.cs.gui4pddl.\textbf{codemodel}} -- ?
\item \texttt{pl.poznan.put.cs.gui4pddl.\textbf{editor}} -- zawiera implementację edytora tekstowego oraz wszystkich jego funkcjonalności -- kolorowania i podpowiadania składni, dopasowywania nawiasów, automatycznych wcięć.
\begin{itemize}
\item \texttt{pl.poznan.put.cs.gui4pddl.\textbf{editor.scanners}} -- zawiera detektory słów kluczowych oraz elementów składniowych języka, które są wykorzystywane na etapie kolorowania i podpowiadania składni.
\end{itemize}
\item \texttt{pl.poznan.put.cs.gui4pddl.\textbf{log}} -- zawiera klasę Log, wykorzystywaną jako mechanizm prostego zgłaszania wyjątków w czasie działania wtyczki. Wyjątki te są widoczne w widoku \textit{Error Log} platformy Eclipse.
\item \texttt{pl.poznan.put.cs.gui4pddl.\textbf{parser}} -- zawiera klasy odpowiedzialne za tworzenie i obsługę analizatora składniowego (parsera) plików PDDL.
\item \texttt{pl.poznan.put.cs.gui4pddl.\textbf{perspective}} -- zawiera definicję perspektywy PDDL, czyli widocznego układu widoków podczas edycji plików PDDL. Charakterystycznymi elementami tej perspektywy są: przeglądarka projektów (ang. \textit{Project Explorer}), edytor plików oraz standardowo widoczna przeglądarka planów (ang. \textit{Plan Browser}).
\item \texttt{pl.poznan.put.cs.gui4pddl.\textbf{planview}} -- pakiet związany z implementacją przeglądarki planów (\textit{Plan Browser}).
\begin{itemize}
\item \texttt{pl.poznan.put.cs.gui4pddl.\textbf{planview.helpers}} -- zawiera klasy pomocnicze, służące do obsługi otwierania plików planów w zewnętrznym edytorze oraz uruchamiania systemowej przeglądarki plików, a także przetwarzania wzorca regularnego nazwy planu, zdefiniowanego w preferencjach plannera (roz.~\ref{subsec:konfiguracja}).
\item \texttt{pl.poznan.put.cs.gui4pddl.\textbf{planview.model}} -- zawiera klasę modelu danych, wykorzystywanych w przeglądarce planów.
\begin{itemize}
\item \texttt{pl.poznan.put.cs.gui4pddl.\textbf{planview.model.manager}} -- zawiera klasy zarządcy danych przeglądarki planów, który umożliwia zapisanie oraz wczytywanie danych z pliku \textit{XML} oraz dodawanie i usuwanie elementów z jednoczesną aktualizacją widoku przeglądarki.
\end{itemize}
\item \texttt{pl.poznan.put.cs.gui4pddl.\textbf{planview.ui}} -- zawiera pełną implementację widoku przeglądarki planów.
\end{itemize}
\item \texttt{pl.poznan.put.cs.gui4pddl.\textbf{preferences}} -- zawiera implementację widoków standardowych stron preferencji wtyczki (preferencje ogólne, edytora oraz plannera), dostępnych z menu głównego \textit{Window} pod elementem \textit{Preferences}.
\begin{itemize}
\item \texttt{pl.poznan.put.cs.gui4pddl.\textbf{preferences.model}} -- zawiera klasę modelu danych preferencji plannera.
\begin{itemize}
\item \texttt{pl.poznan.put.cs.gui4pddl.\textbf{preferences.model.manager}} -- zawiera klasę zarządcy danych preferencji plannera. Pozwala na zapis i odczyt danych z pliku w formacie \textit{XML}.
\end{itemize}
\item \texttt{pl.poznan.put.cs.gui4pddl.\textbf{preferences.ui}} -- zawiera implementację niestandardowych widoków preferencji (tzn. takich, które nie korzystają z \textit{API} \textit{Preferences}) -- widok preferencji plannera oraz jego okna dialogowe.
\end{itemize}
\item \texttt{pl.poznan.put.cs.gui4pddl.\textbf{runners}} -- zawiera klasy odpowiedzialne za wywołanie plannera, od momentu wciśnięcia przycisku \textit{Run}, poprzez stworzenie nowej konfiguracji uruchamiania, aż do inicjalizacji procesu zewnętrznego narzędzia. W pakiecie tym znajduje się także implementacja przerywania procesu znajdowania planu opisana w rozdziale~\ref{subsec:przerywanie}.
\begin{itemize}
\item \texttt{pl.poznan.put.cs.gui4pddl.\textbf{runners.helpers}} -- zawiera pomocniczą klasę (\textit{LaunchUtil}) statycznych metod, związanych głównie ze zwracaniem określonego rodzaju zasobów w projekcie.
\item \texttt{pl.poznan.put.cs.gui4pddl.\textbf{runners.ui}} -- zawiera klasy, związane z tworzeniem okna nowej konfiguracji uruchamiania, dostępnego podczas inicjalizacji zewnętrznego narzędzia lub poprzez wybranie z menu \textit{Run} elementu \textit{Run Configurations...} i przejście na zakładkę \textit{PDDL}.
\begin{itemize}
\item \texttt{pl.poznan.put.cs.gui4pddl.\textbf{runners.ui.blocks}} -- zawiera klasy odpowiedzialne za elementy widoku (bloki) okna nowej konfiguracji uruchamiania.
\end{itemize}
\end{itemize}
\item \texttt{pl.poznan.put.cs.gui4pddl.\textbf{wizards}} -- pakiet związany z implementacją dostępnych we wtyczce kreatorów.
\begin{itemize}
\item \texttt{pl.poznan.put.cs.gui4pddl.\textbf{wizards.ui}} -- zawiera implementację kreatorów tworzenia nowego projektu PDDL oraz nowego pliku domeny lub problemu.
\end{itemize}
\end{itemize}