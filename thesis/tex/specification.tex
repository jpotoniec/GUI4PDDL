\chapter{Specyfikacja wymagań}
\label{sec:specyfikacja}
\section{Wymagania funkcjonalne}
\subsection{Wstęp}
Założeniem projektu było utworzenie lekkiego i wydajnego nardzędzia do programowania w języku PDDL. Dodatkowym ważnym wymaganiem było wsparcie zewnętrznego planera FastDownward oraz możliwe dalsze rozwijanie aplikacji w kiedunku innych planerów. Aplikacja powinna pomagać użytkownikowi w utworzeniu plików PDDL, a później w zarządzaniu kodem. Kluczowym elementem zarządzania kodem jest wykrywanie błędów oraz wyświetlenie kodu w czytelnej formie. Aplikacja ma za zadanie pokolorować kod, oddzielić kluczowe fragmenty i podpowiadać słowa kluczowe, wzorem z popularnych środowisk pracy typu edytory języków Java lub Pascal. Ponadto sama aplikacja miała być łatwo dostępna i łatwa w instalacji, jak również w użytkowaniu.
\JP{A nie możnaby kolorowania i wykrywania błędów połączyć w jedno? Coś pokroju: Wspomaganie użytkownika w pracy z edytorem?}
\subsection{Przypadki użycia}
\subsubsection{Kolorowanie składni}
\begin{tabular}{|p{\textwidth}|}
\hline 
\textbf{Przypadek użycia:} Kolorowanie składni\\
\hline
\textbf{Aktorzy:} Użytkownik\\
\hline
\textbf{Pre:} Użytkownik otwiera pusty dokument\\
\hline
\textbf{Post:} System wyświetla pokolorowany kod\\
\hline
\textbf{Scenariusz główny}\\
\hline
\begin{enumerate}
\item Użytkownik wpisuje ciąg znaków do edytora tekstu.
\item Edytor rozpoznaje typ ciągu znaków.
\item Tekst zostaje pokolorowany na odpowiedni kolor.
\end{enumerate}\\
\hline
\textbf{Rozszerzenia}\\
\hline
\begin{enumerate}
\item[2.a] Ciąg znaków to komentarz.
\item[2.b] Ciąg znaków nie jest komentarzem.
\item[2.b.1] Następuje głębsza analiza znaków.
\end{enumerate}\\
\hline
\end{tabular} 

\subsubsection{Wykrywanie błędów}
\label{reqErrorDetection}
\begin{tabular}{|p{\textwidth}|}
\hline 
\textbf{Przypadek użycia:} Wykrywanie błędów\\
\hline
\textbf{Aktorzy:} Użytkownik\\
\hline
\textbf{Pre:} Użytkownik wpisuje fragment kodu\\
\hline
\textbf{Post:} System wykrywa i wskazuje błędy w dokumencie\\
\hline
\textbf{Scenariusz główny}\\
\hline
\begin{enumerate}
\item Użytkownik wpisuje kod.
\item Edytor pobiera aktualny stan dokumentu.
\item System sprawdza kod pod kątem gramatycznym i semantycznym.
\end{enumerate}\\
\hline
\\\textbf{Rozszerzenia}\\
\hline
\begin{enumerate}
\item[3.a] System wykrywa błąd. 
\item[3.a.1] W domyślnym oknie błędów pojawia się informacja o błędzie i miejscu.
\item[3.a.2] W oknie edytora błąd zostaje podkreślony.
\item[3.b] System nie wykrywa błędu.
\end{enumerate}\\
\hline
\end{tabular}

\subsubsection{Podpowiadanie składni}
\label{reqAutocompletion}
\begin{tabular}{|p{\textwidth}|}
\hline 
\textbf{Przypadek użycia:} Podpowiadanie składni\\
\hline
\textbf{Aktorzy:} Użytkownik\\
\hline
\textbf{Pre:} Użytkownik wpisuje ciąg znaków\\
\hline
\textbf{Post:} System pokazuje użytkownikowi możliwe zakończenia podanego ciągu\\
\hline
\textbf{Scenariusz główny}\\
\hline
\begin{enumerate}
\item Użytkownik podaje ciąg znaków lub korzysta z okna podpowiedzi.
\item Edytor pobiera ostatnią sekwencję znaków przed kursorem.
\item Ciąg znaków poddany jest analizie.
\item Edytor wyświetla listę w oknie podpowiedzi.
\item Użytkownik zaznacza wybraną podpowiedź.
\item Edytor wpisuje tekst za kursorem.
\end{enumerate}\\
\hline
\\\textbf{Rozszerzenia}\\
\hline
\begin{enumerate}
\item[3.a] System zwraca listę propozycji zaczynających się na podany ciąg znaków.
\item[3.b] System nie znajduje podpowiedzi i zwraca pustą listę.
\end{enumerate}\\
\hline
\end{tabular}

\subsubsection{Tworzenie nowego pliku PDDL}
\label{reqAutocompletion}
\begin{tabular}{|p{\textwidth}|}
\hline 
\textbf{Przypadek użycia:} Tworzenie nowego pliku PDDL\\
\hline
\textbf{Aktorzy:} Użytkownik\\
\hline
\textbf{Pre:} Użytkownik chce utworzyć plik PDDL\\
\hline
\textbf{Post:} System otwiera plik z szablonem gotowy do pracy\\
\hline
\textbf{Scenariusz główny}\\
\hline
\begin{enumerate}
\item Użytkownik uruchamia kreatora tworzenia pliku PDDL.
\item System prosi o podanie nazwy pliku.
\item Użytkownik wybiera przeznaczenie pliku.
\item Kreator tworzy nowy plik PDDL z szablonem zgodnym z przeznaczeniem pliku.
\end{enumerate}\\
\hline
\\\textbf{Rozszerzenia}\\
\hline
\begin{enumerate}
\item[3.a] Użytkownik uruchamia kreatora tworzenia niestandardowego pliku.
\item[3.a.1] Użytkownik wybiera typ pliku PDDL z listy propozycji.
\end{enumerate}\\
\hline
\end{tabular}

\subsubsection{Instalacja aplikacji przez repozytorium}
\label{reqAutocompletion}
\begin{tabular}{|p{\textwidth}|}
\hline 
\textbf{Przypadek użycia:} Instalacja aplikacji przez repozytorium\\
\hline
\textbf{Aktorzy:} Użytkownik\\
\hline
\textbf{Pre:} Użytkownik chce zainstalować aplikację\\
\hline
\textbf{Post:} System instaluje aplikację\\
\hline
\textbf{Scenariusz główny}\\
\hline
\begin{enumerate}
\item Użytkownik otwiera okno instalacji dodatkowego oprogramowania Eclipse.
\item Użytkownik wpisuje adres repozytorium.
\item System pokazuje listę dostępnych dodatków z podanego źródła.
\item Użytkownik zaznacza wtyczkę GUI4PDDL do instalacji.
\item System rozpoczyna pobieranie oraz instalację.
\item Użytkownik uruchamia ponownie program Eclipse by ukończyć instalację.
\item Po otwarciu programu ponownie wtyczka widoczna jest na liście aktywnych dodatków Eclipse.
\end{enumerate}\\
\hline
\end{tabular}

\section{Wymagania pozafunkcjonalne}
\subsection*{Funkcjonalność}
Wtyczka \emph{GUI4PDDL} umożliwia użytkownikowi efektywne zarządzanie kodem pisanym w języku \emph{PDDL}. Pomaga w kontrolowaniu zawartości kodu. Wtyczka zbudowana jest jako jeden produkt, zgodnie ze specyfikacją wymaganą przez wtyczki programu \emph{Eclipse}. Zgodnie z polityką twórców Eclipse, \emph{GUI4PDDL} jest zbudowana w pełni modułowo, co umożliwia pełną integrację ze środowiskiem oraz jest łatwa w rozwoju.
\subsection*{Bezpieczeństwo}
Żadne dane nie są wysyłane poza środowisko, dzięki czemu spełnia warunki bezpieczeństwa.
\subsection*{Niezawodność}
Oprogramowanie jest kompletne, dojrzałe. Aplikacja pozbawiona jest błędów, które mogłyby wpłynąć pośrednio lub bezpośrednio na pracę zarówno środowiska programistycznego \emph{Eclipse}, jak i kodu napisanego przez użytkownika. Zdolność do odtworzenia oraz niezawodność zapewnia samo środowisko \emph{Eclipse}, pozwalając na automatyczne odtworzenie kodu z kopii zapasowych w przypadku nieprawidłowego zamknięcia programu.
\subsection*{Użyteczność}
Plugin \emph{GUI4PDDL} wykorzystuje wszelkie formy pomocy dla użytkownika dostępne w środowisku \emph{Eclipse}. Użytkownik jest zawsze w prosty i czytelny sposób poinformowany o stanie swojego kodu oraz o błędach w kodzie. Kolorowanie składni pozwala z łatwością wykryć podstawowe błędy gramatyczne, a automatyczne wcięcia i kolorowe nawiasy umożliwiają szybkie znalezienie braków. Dodatkowo umożliwia intuicyjne korzystanie z wcześniej zdefiniowanych plannerów, wykorzystując np. domyślny przycisk uruchamiania kodu \emph{Eclipse}, co pozwala na szybkie i proste użytkowanie kodu przez użytkownika.
\subsection*{Wydajność}
Aplikacja jest bardzo lekka i pracuje w czasie rzeczywistym. Dzięki multiplatformowości użytkownik może korzystać ze wszystkich funkcji programu w systemach operacyjnych \emph{Windows XP}\ i wyżej oraz we wszystkich dystrybucjach linuksa nowszych od \emph{Ubuntu 10.04} lub porównywalnych w których zainstalowane jest środowisko \emph{JDK}.
\subsection*{Łatwość konserwacji}
Kod jest utrzymywany w postaci modułowej. Modułowość wzorowana na innych wtyczkach programu Eclipse pozwala na łatwą i nieograniczoną rozbudowę aplikacji \emph{GUI4PDDL}. Dzięki architekturze modułowej, w przypadku wystąpienia błędów podczas rozszerzania wtyczki, aplikacja nie narusza ważnych elementów środowiska w którym pracuje. Wykorzystując schematy stosowane przez twórców \emph{Eclipse} kod jest stabilny i czytelny, zachowując odpowiednie standardy.


