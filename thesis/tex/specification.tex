\chapter{Specyfikacja wymagań}
\section{Wymagania funkcjonalne}
\subsection{Kolorowanie składni}
\begin{enumerate}
\item Użytkownik wpisuje ciąg znaków do edytora tekstu.
\item Edytor rozpoznaje typ ciągu znaków.
\item Tekst zostaje pokolorowany na odpowiedni kolor.
\end{enumerate}
Rozszerzenia
\begin{enumerate}
\item[2.a] Ciąg znaków to komentarz.
\item[2.b] Ciąg znaków nie jest komentarzem.
\item[2.b.1] Następuje głębsza analiza znaków.
\end{enumerate}
\subsection{Wykrywanie błędów}
\begin{enumerate}
\item Użytkownik wpisuje kod.
\item Edytor pobiera aktualny stan dokumentu.
\item System sprawdza kod pod kątem gramatycznym i semantycznym.
\end{enumerate}
Rozszerzenia
\begin{enumerate}
\item[3.a] System wykrywa błąd. 
\item[3.a.1] W domyślnym oknie błędów pojawia się informacja o błędzie i miejscu.
\item[3.a.2] W oknie edytora błąd zostaje podkreślony.
\item[3.b] System nie wykrywa błędu.
\end{enumerate}
\subsection{Podpowiadanie składni}
\begin{enumerate}
\item Użytkownik podaje ciąg znaków lub korzysta z okna podpowiedzi.
\item Edytor pobiera ostatnią sekwencję znaków przed kursorem.
\item Ciąg znaków poddany jest analizie.
\item Edytor wyświetla listę w oknie podpowiedzi.
\item Użytkownik zaznacza wybraną podpowiedź.
\item Edytor wpisuje tekst za kursorem.
\end{enumerate}
Rozszerzenia
\begin{enumerate}
\item[3.a] System zwraca listę propozycji zaczynających się na podany ciąg znaków.
\item[3.b] System nie znajduje podpowiedzi i zwraca pustą listę.
\end{enumerate}
\subsection{Tworzenie nowego projektu}
\begin{enumerate}
\item Użytkownik otwiera menu kontekstowe.
\item Użytkownik wybiera opcję nowy.
\item Użytkownik uruchamia kreatora projektu PDDL.
\item Użytkownik proszony jest o podanie nazwy projektu
\item Kreator tworzy nowy, pusty projekt.
\end{enumerate}
Rozszerzenia
\begin{enumerate}
\item[3.a] Użytkownik uruchamia kreatora tworzenia nowego projektu. 
\item[3.a.1] Użytkownik wybiera projekt PDDL z listy propozycji.
\end{enumerate}
\subsection{Tworzenie nowego pliku PDDL}
\begin{enumerate}
\item Użytkownik otwiera menu kontekstowe.
\item Użytkownik wybiera opcję nowy.
\item Użytkownik uruchamia kreatora tworzenia pliku PDDL.
\item Użytkownik proszony jest o podanie nazwy pliku.
\item Użytkownik wybiera przeznaczenie pliku.
\item Kreator tworzy nowy plik PDDL z szablonem zgodnym z przeznaczeniem pliku.
\end{enumerate}
Rozszerzenia
\begin{enumerate}
\item[3.a] Użytkownik uruchamia kreatora tworzenia niestandardowego pliku.
\item[3.a.1] Użytkownik wybiera typ pliku PDDL z listy propozycji.
\end{enumerate}
\subsection{Repozytorium kodu}
\begin{enumerate}
\item Użytkownik otwiera okno instalacji dodatkowego oprogramowania Eclipse.
\item Użytkownik wpisuje adres repozytorium.
\item System pokazuje listę dostępnych dodatków z podanego źródła.
\item Użytkownik zaznacza wtyczkę GUI4PDDL do instalacji.
\item System rozpoczyna pobieranie oraz instalację.
\item Użytkownik uruchamia ponownie program Eclipse by ukończyć instalację.
\item Po otwarciu programu ponownie wtyczka widoczna jest na liście aktywnych dodatków Eclipse.
\end{enumerate}
\section{Wymagania pozafunkcjonalne}


